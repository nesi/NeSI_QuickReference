\documentclass[final,t]{beamer}
\mode<presentation>
{
  \usetheme{NeSI}
%  \usetheme{Aachen}
%  \usetheme{Oldi6}
%  \usetheme{I6td}
%  \usetheme{I6dv}
%  \usetheme{I6pd}
%  \usetheme{I6pd2}
}
% additional settings
\setbeamerfont{itemize}{size=\normalsize}
\setbeamerfont{itemize/enumerate body}{size=\normalsize}
\setbeamerfont{itemize/enumerate subbody}{size=\normalsize}

% additional packages
\usepackage{times}
\usepackage{amsmath, amsthm, amssymb, soul, color, multicol, type1cm, verbatim, latexsym, float,multicol}
\usepackage{exscale}
%\boldmath
\usepackage{booktabs, array}
%\usepackage{rotating} %sideways environment
\usepackage[english]{babel}
\usepackage[latin1]{inputenc}
%\usepackage[orientation=landscape,size=custom,width=200,height=120,scale=1.9]{beamerposter}
\usepackage[orientation=landscape,size=a4]{beamerposter}
\listfiles
\graphicspath{{figures/}}
% Display a grid to help align images
%\beamertemplategridbackground[1cm]

\usepackage{listings}
\usepackage{xspace}
\usepackage{fp}
\usepackage{ifthen}

\title{\huge High Performance Computing Quick Reference Guide to the NeSI Pan Cluster}
\author{New Zealand eScience Infrastructure (\href{https://www.nesi.org.nz/}{NeSI}) @ \href{http://www.eresearch.auckland.ac.nz/}{The University of Auckland}}
\institute[]{
NeSI is a partnership for all New Zealand researchers delivered by 
  \href{http://www.eresearch.auckland.ac.nz/}{The University of Auckland}, \href{http://www.niwa.co.nz/our-services/hpcf}{NIWA}, \href{http://www.bluefern.canterbury.ac.nz/}{University of Canterbury}, \\
  \href{http://www.landcareresearch.co.nz/}{Landcare Research}, \href{http://www.otago.ac.nz/}{Otago University} and \href{http://msi.govt.nz/}{MBIE}
}

% abbreviations
\makeatletter
\makeatother

%%%%%%%%%%%%%%%%%%%%%%%%%%%%%%%%%%%%%%%%%%%%%%%%%%%%%%%%%%%%%%%%%%%%%%%%%%%%%%%%%%%%%%%%%%%%%%%%%%%%%%%%%%%%
%%%%%%%%%%%%%%%%%%%%%%%%%%%%%%%%%%%%%%%%%%%%%%%%%%%%%%%%%%%%%%%%%%%%%%%%%%%%%%%%%%%%%%%%%%%%%%%%%%%%%%%%%%%%
\begin{document}
\begin{frame}[fragile]{} 
  \begin{columns}[t]
    \begin{column}{.32\linewidth}

      %%%%%%%%%%%%%%%%%%%%%%%%%%%%%%%%%%%%%%%%%%%%%%%%%%%%%%%%%%%%%%%%%%%%%%%%%%%%%%%%%%%%%%%%%%%%%%%%%%%%%%%%%%%%

      \begin{block}{Contact Information}
        \begin{itemize}
        \item \textbf{web}: \url{http://www.nesi.org.nz}
        \item \textbf{wiki}: \url{https://wiki.auckland.ac.nz/display/CER/}
        \item \textbf{ganglia}: \url{http://ganglia.uoa.nesi.org.nz}
        \item \textbf{support} : \href{mailto:support@nesi.org.nz}{support@nesi.org.nz}
        \end{itemize}
        \vspace*{-1.5cm}
        \hspace*{7cm}
        \includegraphics[width=1.4cm]{img/qrcode.png} 
      \end{block}

      %%%%%%%%%%%%%%%%%%%%%%%%%%%%%%%%%%%%%%%%%%%%%%%%%%%%%%%%%%%%%%%%%%%%%%%%%%%%%%%%%%%%%%%%%%%%%%%%%%%%%%%%%%%%
      \begin{block}{Useful Quick References}
        \vspace*{-3ex}
        \begin{multicols}{2}
        \begin{itemize}
        \item \href{https://wiki.auckland.ac.nz/download/attachments/37950883/vi+Quick+Reference.pdf}{VI Quick Reference}
        \item \href{https://wiki.auckland.ac.nz/download/attachments/37950883/Bash+Quick+Reference.pdf}{BASH Quick Reference} 
        \item \href{https://wiki.auckland.ac.nz}{Linux Quick Reference}
        \item \href{https://wiki.auckland.ac.nz/download/attachments/37950883/OpenMP3.1-FortranCard.pdf}{OpenMP Fortran Syntax}
        \item \href{https://wiki.auckland.ac.nz/download/attachments/37950883/OpenMP3.1-CCard.pdf}{OpenMP 3.1 API C/C++ Syntax}
        \item \href{https://wiki.auckland.ac.nz}{MPI Quick Reference}
        \end{itemize}
        \end{multicols}
        \vspace*{-3ex}
      \end{block}
      %%%%%%%%%%%%%%%%%%%%%%%%%%%%%%%%%%%%%%%%%%%%%%%%%%%%%%%%%%%%%%%%%%%%%%%%%%%%%%%%%%%%%%%%%%%%%%%%%%%%%%%%%%%%

      %%%%%%%%%%%%%%%%%%%%%%%%%%%%%%%%%%%%%%%%%%%%%%%%%%%%%%%%%%%%%%%%%%%%%%%%%%%%%%%%%%%%%%%%%%%%%%%%%%%%%%%%%%%%
      \begin{block}{NeSI Pan Cluster}
        \vspace*{-2ex}
      \begin{center}
      \begin{tabular}{|c|c|c|c|}
      \hline 
      \textbf{Architecture} & \textbf{Westmere} & \textbf{SandyBridge} & \textbf{LargeMem} \\ 
      \hline 
      Model & X5660 & E5-2680 & E7-4870 \\ 
      \hline 
      Clock Speed & 2.8 GHz & 2.7 GHz & 2.4GHz \\ 
      \hline 
      Cache & 12MB & 20MB & 30MB \\ 
      \hline 
      Intel QPI speed & 6.4GT/s & 8 GT/s & 6.4GT/ \\ 
      \hline 
      Cores/socket & 6 & 8 & 10 \\ 
      \hline       
      Cores/node & 12 & 16 & 40 \\ 
      \hline 
      Mem/node & 96GB & 128GB & 512GB \\ 
      \hline 
      GFLOPS/node & 134.4 & 345.6 & 384.0 \\ 
      \hline 
      \# nodes & 76 & 194 & 4 \\ 
      \hline 
      \end{tabular} 
      \end{center}
        \vspace*{-2ex}
      \end{block}
      %%%%%%%%%%%%%%%%%%%%%%%%%%%%%%%%%%%%%%%%%%%%%%%%%%%%%%%%%%%%%%%%%%%%%%%%%%%%%%%%%%%%%%%%%%%%%%%%%%%%%%%%%%%%


      %%%%%%%%%%%%%%%%%%%%%%%%%%%%%%%%%%%%%%%%%%%%%%%%%%%%%%%%%%%%%%%%%%%%%%%%%%%%%%%%%%%%%%%%%%%%%%%%%%%%%%%%%%%%
      \begin{block}{NeSI Pan Cluster - Co-Processors}
        \vspace*{-2ex}
      \begin{center}
      \begin{tabular}{|c|c|c|c|}
      \hline 
      \textbf{Architecture} & \textbf{Nvidia Fermi} & \textbf{Nvidia Kepler} & \textbf{Intel Phi} \\ 
      \hline 
      Main CPU                  &  X5660/E5-2680       &  E5-2680                   &  E5-2680 \\
      \hline 
      Model                       &  M2090                     &   K20X                       &  5110P \\ 
      \hline 
      Clock Speed             &  1.3GHz                    &  0.732GHz                      &  1.053GHz\\ 
      \hline 
      Cores/Dev.             & 512                            & 2688                           & 60 (240) \\ 
      \hline       
      Dev./node              &  2                             &   2                                &  2  \\ 
      \hline 
      Mem/Dev.             & 6GB                          & 6GB                               & 8GB  \\ 
      \hline 
      TFLOPS/Dev         & 1.33                          & 1.17                              & 1.01 \\ 
      \hline 
      \# nodes               & 16                             & 5                                 & 2 \\ 
      \hline 
      \end{tabular} 
      \end{center}
        \vspace*{-2ex}
      \end{block}
      %%%%%%%%%%%%%%%%%%%%%%%%%%%%%%%%%%%%%%%%%%%%%%%%%%%%%%%%%%%%%%%%%%%%%%%%%%%%%%%%%%%%%%%%%%%%%%%%%%%%%%%%%%%%


      %%%%%%%%%%%%%%%%%%%%%%%%%%%%%%%%%%%%%%%%%%%%%%%%%%%%%%%%%%%%%%%%%%%%%%%%%%%%%%%%%%%%%%%%%%%%%%%%%%%%%%%%%%%%
      \begin{block}{Disk Spaces + Default Quota}
        \vspace*{-2ex}
      \begin{center}
      \begin{tabular}{|l|c|c|c|c|c|c|}
      \hline 
      \textbf{FileSystem} & \textbf{Space} & \textbf{Quota} & \textbf{ACL} & \textbf{Backup} & \textbf{Type} & \textbf{Usage}  \\ 
      \hline 
      \$HOME & 120TB & 30GB & rw & yes & GPFS & Archive \\ 
      \hline 
      /share & 120TB & - & ro & yes & GPFS & Archive \\ 
      \hline 
      \$TMP\_DIR & 240GB & - & rw & NO & EXT4 & io \\ 
      \hline 
      \$SCRATCH\_DIR & 8.8TB & - & rw & NO & GPFS & io \\ 
      \hline 
      \$SHM\_DIR & Mem & - & rw & NO & RamFS & io \\ 
      \hline 
      \end{tabular} 
      \end{center}
        \vspace*{-2ex}
      \end{block}
      %%%%%%%%%%%%%%%%%%%%%%%%%%%%%%%%%%%%%%%%%%%%%%%%%%%%%%%%%%%%%%%%%%%%%%%%%%%%%%%%%%%%%%%%%%%%%%%%%%%%%%%%%%%%
 
 


 

     \end{column}

%%%%%%%%%%%%%%%%%%%%%%%%%%%%%%%%%%%%%%%%%%%%%%%%%%%%%%%%%%%%%%
%%%%%%%%%%%%%%%%%%%%%%%%%%%%%%%%%%%%%%%%%%%%%%%%%%%%%%%%%%%%%%
%%%%%%%%%%%%%%%%%%%%%%%%%%%%%%%%%%%%%%%%%%%%%%%%%%%%%%%%%%%%%%
    
    \begin{column}{.32\linewidth}
    \vskip-2ex
    \begin{columns}[t]
    \begin{column}{.54\linewidth}
      %%%%%%%%%%%%%%%%%%%%%%%%%%%%%%%%%%%%%%%%%%%%%%%%%%%%%%%%%%%%%%%%%%%%%%%%%%%%%%%%%%%%%%%%%%%%%%%%%%%%%%%%%%%%
      \begin{block}{Available Compilers}
        \vspace*{-2ex}
      \begin{center}
      \begin{tabular}{|c|c|c|c|}
      \hline 
      \textbf{Compiler} & \textbf{Intel} & \textbf{GNU} & \textbf{PGI} \\ 
      \hline 
      Fortran77 & ifort & g77 & pgf77 \\ 
      \hline 
      Fortran90 & ifort & gfortran & pgf90 \\ 
      \hline 
      Fortran95 & ifort & gfortran & pgf95 \\ 
      \hline 
      C & icc & gcc & pgcc \\ 
      \hline 
      C++ & icpc & g++ & pgCC \\ 
      \hline 
      Debug & idb & gdb & pgdbg \\ 
      \hline
      Profile & vtune & gprof & gpprof \\ 
      \hline 
      \end{tabular} 
      \end{center}
        \vspace*{-2ex}
      \end{block}
      %%%%%%%%%%%%%%%%%%%%%%%%%%%%%%%%%%%%%%%%%%%%%%%%%%%%%%%%%%%%%%%%%%%%%%%%%%%%%%%%%%%%%%%%%%%%%%%%%%%%%%%%%%%%
      \end{column}
     \hspace*{-0.3cm}
     \begin{column}{.35\linewidth}
      %%%%%%%%%%%%%%%%%%%%%%%%%%%%%%%%%%%%%%%%%%%%%%%%%%%%%%%%%%%%%%%%%%%%%%%%%%%%%%%%%%%%%%%%%%%%%%%%%%%%%%%%%%%%
      \begin{block}{Available MPIs}
        \vspace*{-2ex}
      \begin{center}
      \begin{tabular}{|c|c|}
      \hline 
      \textbf{MPI} & \textbf{version} \\ 
      \hline 
      Intel MPI & 4.1.0.024  \\ 
      \hline 
      OpenMPI & 1.4,1.6  \\ 
      \hline 
      MPICH2 & 1.5,3.0.4  \\ 
      \hline 
      PlatformMPI & 08.02  \\ 
      \hline 
      MVAPICH2 & 1.4.1p1  \\ 
      \hline 
      \end{tabular} 
      \end{center}
        \vspace*{-2ex}
      \end{block}
      %%%%%%%%%%%%%%%%%%%%%%%%%%%%%%%%%%%%%%%%%%%%%%%%%%%%%%%%%%%%%%%%%%%%%%%%%%%%%%%%%%%%%%%%%%%%%%%%%%%%%%%%%%%%
      \end{column}
  \end{columns}

 
       %%%%%%%%%%%%%%%%%%%%%%%%%%%%%%%%%%%%%%%%%%%%%%%%%%%%%%%%%%%%%%%%%%%%%%%%%%%%%%%%%%%%%%%%%%%%%%%%%%%%%%%%%%%%
      \begin{block}{Optimization flags}
        \vspace*{-2ex}
           \begin{center}
           \begin{tabular}{|c|c|c|c|}
           \hline 
           \textbf{Compiler} & \textbf{Intel} & \textbf{GNU} & \textbf{PGI} \\ 
           \hline 
           High Opt. & -fast & -O3 -ffast-math & -fast -Mipa=fast,inline \\ 
           \hline 
           OpenMP & -openmp & -fopenmp & -mp=nonuma \\ 
           \hline 
           Debug & -g & -g & -g \\ 
           \hline 
           Profile & -p & -pg & -p \\ 
           \hline 
           Westmere & -mtarget & -march=corei7  & -tp=nehalem-64 \\ 
           \hline 
           SandyBridge & -mtarget & -march=corei7-avx  & -tp=sandybridge-64 \\ 
           \hline 
           SSE & -xsse4.2 & -msse4.2 & -Mvect=[prefetch,sse] \\ 
           \hline 
           AVX$^1$ & -xavx & -mavx & -fast \\ 
           \hline 
           \end{tabular} 
           \begin{footnotesize}
            \\1. Advanced Vector Extension (AVX) streaming SIMD instructions. Sandy Bridge processor only.
            %2. for SandyBridge use : -march=corei7-avx\\
           \end{footnotesize}
           \end{center}
        \vspace*{-2ex}
      \end{block}
      %%%%%%%%%%%%%%%%%%%%%%%%%%%%%%%%%%%%%%%%%%%%%%%%%%%%%%%%%%%%%%%%%%%%%%%%%%%%%%%%%%%%%%%%%%%%%%%%%%%%%%%%%%%%
     
        %%%%%%%%%%%%%%%%%%%%%%%%%%%%%%%%%%%%%%%%%%%%%%%%%%%%%%%%%%%%%%%%%%%%%%%%%%%%%%%%%%%%%%%%%%%%%%%%%%%%%%%%%%%%
      \begin{block}{Link MKL with OpenMPI, CDFT, ScaLAPACK, BLACS and Intel Compilers}
      \textbf{Link line:} \verb|-L${MKLROOT}/lib/intel64 -lmkl_scalapack_ilp64 \|
      \verb|        -lmkl_cdft_core -lmkl_intel_ilp64 -lmkl_sequential \|
      \verb|        -lmkl_core -lmkl_blacs_intelmpi_ilp64 -lpthread -lm|\\
      \textbf{Compiler options: }      \verb|-DMKL_ILP64 -I${MKLROOT}/include|\\
      More information at \url{http://software.intel.com/sites/products/mkl/}
      \end{block}
      %%%%%%%%%%%%%%%%%%%%%%%%%%%%%%%%%%%%%%%%%%%%%%%%%%%%%%%%%%%%%%%%%%%%%%%%%%%%%%%%%%%%%%%%%%%%%%%%%%%%%%%%%%%%
 
 
         %%%%%%%%%%%%%%%%%%%%%%%%%%%%%%%%%%%%%%%%%%%%%%%%%%%%%%%%%%%%%%%%%%%%%%%%%%%%%%%%%%%%%%%%%%%%%%%%%%%%%%%%%%%%
      \begin{block}{Link MKL with OpenMP and Intel compilers}
      \textbf{Link line:} \verb|-L${MKLROOT}/lib/intel64 -lmkl_intel_ilp64 \|\\
      \verb|       -lmkl_intel_thread -lmkl_core -lpthread -lm|\\
      \textbf{Compiler options: } \verb|-openmp -DMKL_ILP64 -I${MKLROOT}/include|
      \end{block}
      %%%%%%%%%%%%%%%%%%%%%%%%%%%%%%%%%%%%%%%%%%%%%%%%%%%%%%%%%%%%%%%%%%%%%%%%%%%%%%%%%%%%%%%%%%%%%%%%%%%%%%%%%%%%
 
 
       %%%%%%%%%%%%%%%%%%%%%%%%%%%%%%%%%%%%%%%%%%%%%%%%%%%%%%%%%%%%%%%%%%%%%%%%%%%%%%%%%%%%%%%%%%%%%%%%%%%%%%%%%%%%
      \begin{block}{Get Involved!}
       \centering{\includegraphics[scale=0.6]{img/avatar_19476.png}\\
       {\Large We would like to encourage you to send suggestions and feedback to the NeSI Team (\url{support@nesi.org.nz}).}}
      \end{block}
      %%%%%%%%%%%%%%%%%%%%%%%%%%%%%%%%%%%%%%%%%%%%%%%%%%%%%%%%%%%%%%%%%%%%%%%%%%%%%%%%%%%%%%%%%%%%%%%%%%%%%%%%%%%%
 
     \end{column}

%%%%%%%%%%%%%%%%%%%%%%%%%%%%%%%%%%%%%%%%%%%%%%%%%%%%%%%%%%%%%%
%%%%%%%%%%%%%%%%%%%%%%%%%%%%%%%%%%%%%%%%%%%%%%%%%%%%%%%%%%%%%%
%%%%%%%%%%%%%%%%%%%%%%%%%%%%%%%%%%%%%%%%%%%%%%%%%%%%%%%%%%%%%%
    
    \begin{column}{.32\linewidth}
    \vskip-2ex
%    \begin{columns}[t]
%    \begin{column}{.38\linewidth}
      %%%%%%%%%%%%%%%%%%%%%%%%%%%%%%%%%%%%%%%%%%%%%%%%%%%%%%%%%%%%%%%%%%%%%%%%%%%%%%%%%%%%%%%%%%%%%%%%%%%%%%%%%%%%
      \begin{block}{OpenSSH Access : Login Node}
      The login node is not for running jobs, it is only for file management and job submission.\\
      \textbf{Parameters}
        \begin{itemize}
        \item host: login-01.uoa.nesi.org.nz
        \item port: 22
        \end{itemize}
      \end{block}
      %%%%%%%%%%%%%%%%%%%%%%%%%%%%%%%%%%%%%%%%%%%%%%%%%%%%%%%%%%%%%%%%%%%%%%%%%%%%%%%%%%%%%%%%%%%%%%%%%%%%%%%%%%%%
%      \end{column}
%     \hspace*{-0.6cm}
%     \begin{column}{.56\linewidth}
     
%      \end{column}
%  \end{columns}

      %%%%%%%%%%%%%%%%%%%%%%%%%%%%%%%%%%%%%%%%%%%%%%%%%%%%%%%%%%%%%%%%%%%%%%%%%%%%%%%%%%%%%%%%%%%%%%%%%%%%%%%%%%%%
      \begin{block}{Suggested Software}
        \begin{itemize}
        \item \textbf{mobaxterm} (Windows) : \url{http://mobaxterm.mobatek.net} 
        \item \textbf{Putty} (Windows) : \url{http://www.chiark.greenend.org.uk/~sgtatham/putty/}
        \item \textbf{Terminal} (MacOSX) :  (Included in the OS)
        \item \textbf{iTerm2} (MacOSX) : \url{http://www.iterm2.com}
        \item \textbf{Konsole} (Linux) : \url{http://konsole.kde.org}
        \item \textbf{GnomeTerminal} (Linux) : \url{https://wiki.gnome.org/Apps/Terminal}
        \item \textbf{yakuake} (Linux) : \url{http://yakuake.kde.org}
        \end{itemize}
      \end{block}
      %%%%%%%%%%%%%%%%%%%%%%%%%%%%%%%%%%%%%%%%%%%%%%%%%%%%%%%%%%%%%%%%%%%%%%%%%%%%%%%%%%%%%%%%%%%%%%%%%%%%%%%%%%%%
      

      %%%%%%%%%%%%%%%%%%%%%%%%%%%%%%%%%%%%%%%%%%%%%%%%%%%%%%%%%%%%%%%%%%%%%%%%%%%%%%%%%%%%%%%%%%%%%%%%%%%%%%%%%%%%
      \begin{block}{Remote File System Access}
       In order to access the file system (/home) remotely from your machine, we recommend:
        \begin{itemize}
        \item \textbf{SSHFS} (MacOSX) : \url{http://code.google.com/p/macfuse/}
        \item \textbf{SSHFS} (Linux) : \url{http://fuse.sourceforge.net/sshfs.html}
        \item \textbf{SSHFS} (Windows) : \url{http://code.google.com/p/win-sshfs/}
        \item \textbf{Konqueror} (KDE) : type fish://user@host:port
        \item \textbf{Nautilus} (Gnome) : type sftp://user@host:port
        \item \textbf{WinSCP} (Windows) : \url{http://winscp.net}
        \end{itemize}
      \end{block}
      %%%%%%%%%%%%%%%%%%%%%%%%%%%%%%%%%%%%%%%%%%%%%%%%%%%%%%%%%%%%%%%%%%%%%%%%%%%%%%%%%%%%%%%%%%%%%%%%%%%%%%%%%%%%
 
       %%%%%%%%%%%%%%%%%%%%%%%%%%%%%%%%%%%%%%%%%%%%%%%%%%%%%%%%%%%%%%%%%%%%%%%%%%%%%%%%%%%%%%%%%%%%%%%%%%%%%%%%%%%%
      \begin{block}{Remote File System Transfer with RSYNC (Unix Only)}
       \textbf{RSYNC} over SSH protocol is the best choice to transfer big data volumes.
        \begin{itemize}
        \item Transfer data from your machine to the server:\\ \verb|rsync -avHl  /path/origin/* sshserver:/path/destination/|
        \item Transfer data from the server to your machine:\\ \verb|rsync -avHl  sshserver:/path/destination/* /path/origin/|
        \end{itemize}
      \end{block}
      %%%%%%%%%%%%%%%%%%%%%%%%%%%%%%%%%%%%%%%%%%%%%%%%%%%%%%%%%%%%%%%%%%%%%%%%%%%%%%%%%%%%%%%%%%%%%%%%%%%%%%%%%%%%

       %%%%%%%%%%%%%%%%%%%%%%%%%%%%%%%%%%%%%%%%%%%%%%%%%%%%%%%%%%%%%%%%%%%%%%%%%%%%%%%%%%%%%%%%%%%%%%%%%%%%%%%%%%%%
      \begin{block}{Remote File System Transfer with scp/sftp (Unix Only)}
       \textbf{SCP} and \textbf{SFTP} are the most popular software to transfer data across the SSH protocol.
        \begin{itemize}
        \item \textbf{SCP}: \verb|scp -pr sshserver:/path/destination/* path/destination/|
        \item \textbf{SFTP}: \verb|sftp sshserver:/path/destination/* path/destination/|
        \end{itemize}
      \end{block}
      %%%%%%%%%%%%%%%%%%%%%%%%%%%%%%%%%%%%%%%%%%%%%%%%%%%%%%%%%%%%%%%%%%%%%%%%%%%%%%%%%%%%%%%%%%%%%%%%%%%%%%%%%%%%

      %%%%%%%%%%%%%%%%%%%%%%%%%%%%%%%%%%%%%%%%%%%%%%%%%%%%%%%%%%%%%%%%%%%%%%%%%%%%%%%%%%%%%%%%%%%%%%%%%%%%%%%%%%%%
      \begin{block}{Recommended Best Practices}
      \begin{itemize}
         \item Use the \textbf{SubmitScripts Templates} that are in /share/SubmitScripts.
         \item Ask for the minimum nodes for the specified number of cores for the very large jobs. It will take more time to enter in execution but will finish with less walltime.
        \end{itemize}
      \end{block}
      %%%%%%%%%%%%%%%%%%%%%%%%%%%%%%%%%%%%%%%%%%%%%%%%%%%%%%%%%%%%%%%%%%%%%%%%%%%%%%%%%%%%%%%%%%%%%%%%%%%%%%%%%%%%
 
    \end{column}
  \end{columns}
\end{frame}
%%%%%%%%%%%%%%%%%%%%%%%%%%%%%%%%%%%%%%%%%%%%%%%%%%%%%%%%%%%%%%
%                                                        NEW FRAME                                                     %
%%%%%%%%%%%%%%%%%%%%%%%%%%%%%%%%%%%%%%%%%%%%%%%%%%%%%%%%%%%%%%
 \begin{frame}[fragile]{} 
  \begin{columns}[t]
    \begin{column}{.35\linewidth}


  %%%%%%%%%%%%%%%%%%%%%%%%%%%%%%%%%%%%%%%%%%%%%%%%%%%%%%%%%%%%%%%%%%%%%%%%%%%%%%%%%%%%%%%%%%%%%%%%%%%%%%%%%%%%
      \begin{block}{Slurm Commands}
        \begin{itemize}
        \item  \textbf{sbatch} - submits a script job.
        \item \textbf{scancel} - cancels a running or pending job.
	    \item \textbf{sinfo} - provides information on partitions and nodes.
	    \item \textbf{sview} - GUI to view job, node and partition information.
	    \item \textbf{smap} - CLI to view job, node and partition information.
        \item  \textbf{squeue} - shows the status of jobs. 
         \begin{itemize}
              \item \verb|squeue -l -j <job id>| gives extended job record information.
              \item \verb|squeue -u <user name>| lists only jobs initiated by the user.
        \end{itemize}
        \item \textbf{sbcast} - transfer file to a compute nodes allocated to a job.
        \item \textbf{interactive} - opens an interactive job session.
	    \item \textbf{sattach} - connects stdin/out/err for an existing job or job step.
        \end{itemize}
       More information about Slurm in the NeSI Slurm User Guide : \url{https://wiki.auckland.ac.nz/display/CER/Slurm+User+Guide}
      \end{block}
      %%%%%%%%%%%%%%%%%%%%%%%%%%%%%%%%%%%%%%%%%%%%%%%%%%%%%%%%%%%%%%%%%%%%%%%%%%%%%%%%%%%%%%%%%%%%%%%%%%%%%%%%%%%%

     %%%%%%%%%%%%%%%%%%%%%%%%%%%%%%%%%%%%%%%%%%%%%%%%%%%%%%%%%%%%%%%%%%%%%%%%%%%%%%%%%%%%%%%%%%%%%%%%%%%%%%%%%%%%
      \begin{block}{Slurm Submit Script Syntax}
        \begin{itemize}
        \item \verb|-A uoa99999| User account (i.e. uoaXXXXX or nesiXXXXX)
        \item \verb|--mem-per-cpu=8132| = memory/cpu (in MB)
        \item \verb|-J My_Job_Name| Sets the job name
        \item \verb|--mail-type=ALL| Specifies under which conditions Slurm will send out email notifications to the specified address about the state of the job:
        \begin{itemize}
            \item ALL - notifies about every change in the job status.
            \item BEGIN - notifies when the job is activated.
            \item END - notifies when the job ends.
            \item FAIL - notifies only if the job fails.
            \item REQUEUE - notifies only if the job is requeued.
        \end{itemize}
        \item \verb|--mail-user=username@nesi.org.nz| user email
        \item \verb|--cpus-per-task=8| Sets the number of cores to be allocated for each task
        \item \verb|--gres=gpu:1| Specifies what consumable resources the job requires to run per task:
        \begin{itemize}
            \item gpu:N - where N is the number of GPU devices
            \item mic:N - where N is the number of MIC devices
        \end{itemize}
        \item \verb|--ntasks-per-node=16| specifies the total number of tasks to be run on each available node.
        \item \verb|--ntasks=96| specifies how many tasks (cores) are to be run in total
        \item \verb|--time=01:00:00| Sets the limit for the elapsed time for which a job can run
        \item \verb|-C sb| Requires an specific hardware architecture (sb=Sandybridge,wm=Westmere)
        \end{itemize}
      \end{block}
      %%%%%%%%%%%%%%%%%%%%%%%%%%%%%%%%%%%%%%%%%%%%%%%%%%%%%%%%%%%%%%%%%%%%%%%%%%%%%%%%%%%%%%%%%%%%%%%%%%%%%%%%%%%%

     %%%%%%%%%%%%%%%%%%%%%%%%%%%%%%%%%%%%%%%%%%%%%%%%%%%%%%%%%%%%%%%%%%%%%%%%%%%%%%%%%%%%%%%%%%%%%%%%%%%%%%%%%%%%
      \begin{block}{Interactive sessions}
      The interactive sessions will allow you to build the binaries for specific architecture. The binaries compiled in Westmere can run in Sandy Bridge, but it can \textbf{NOT} exploit all the Sandy bridge features. The binaries compiled in Sandy Bridge can \textbf{NOT} run in the Westmere nodes. The interactively usage is limited up to \textbf{24h} of \textbf{walltime}.
	 \begin{verbatim}
Usage: interactive [-A] [-a] [-c] [-m] [-J]
Mandatory arguments:
	 -A: account
Optional arguments:
	 -a: architecture (default: wm, values sb=SandyBridge wm=Westmere)
	 -c: number of CPU cores (default: 1)
	 -m: amount of memory (GB) per core (default: 1 [GB])
	 -J: job name
example : interactive -A nesi99999 -a wm -c 4 -J MyInteractiveJob
	\end{verbatim} 
      \end{block}
      %%%%%%%%%%%%%%%%%%%%%%%%%%%%%%%%%%%%%%%%%%%%%%%%%%%%%%%%%%%%%%%%%%%%%%%%%%%%%%%%%%%%%%%%%%%%%%%%%%%%%%%%%%%%


    \end{column}

%%%%%%%%%%%%%%%%%%%%%%%%%%%%%%%%%%%%%%%%%%%%%%%%%%%%%%%%%%%%%%
%%%%%%%%%%%%%%%%%%%%%%%%%%%%%%%%%%%%%%%%%%%%%%%%%%%%%%%%%%%%%%
%%%%%%%%%%%%%%%%%%%%%%%%%%%%%%%%%%%%%%%%%%%%%%%%%%%%%%%%%%%%%%
    
    \begin{column}{.3\linewidth}
    

      %%%%%%%%%%%%%%%%%%%%%%%%%%%%%%%%%%%%%%%%%%%%%%%%%%%%%%%%%%%%%%%%%%%%%%%%%%%%%%%%%%%%%%%%%%%%%%%%%%%%%%%%%%%%
      \begin{block}{Submit Script Example : Serial}
              \vspace*{-3ex}
        \begin{verbatim}
#!/bin/bash
#SBATCH -J OpenMP_JOB
#SBATCH -A uoa99999         # Project Account
#SBATCH --time=01:00:00     # Walltime
#SBATCH --mem-per-cpu=8132  # memory/cpu (in MB)
#SBATCH -C sb               # sb=Sandybridge,wm=Westmere
srun serial_binary
        \end{verbatim}
                \vspace*{-4ex}
      \end{block}
      %%%%%%%%%%%%%%%%%%%%%%%%%%%%%%%%%%%%%%%%%%%%%%%%%%%%%%%%%%%%%%%%%%%%%%%%%%%%%%%%%%%%%%%%%%%%%%%%%%%%%%%%%%%%

      %%%%%%%%%%%%%%%%%%%%%%%%%%%%%%%%%%%%%%%%%%%%%%%%%%%%%%%%%%%%%%%%%%%%%%%%%%%%%%%%%%%%%%%%%%%%%%%%%%%%%%%%%%%%
      \begin{block}{Submit Script Example : OpenMP}
              \vspace*{-3ex}
        \begin{verbatim}
#!/bin/bash
#SBATCH -J OpenMP_JOB
#SBATCH -A uoa99999         # Project Account
#SBATCH --time=01:00:00     # Walltime
#SBATCH --mem-per-cpu=8132  # memory/cpu (in MB)
#SBATCH --cpus-per-task=8   # 8 OpenMP Threads
#SBATCH -C sb               # sb=Sandybridge,wm=Westmere
srun openmp_binary
        \end{verbatim}
                \vspace*{-4ex}
      \end{block}
      %%%%%%%%%%%%%%%%%%%%%%%%%%%%%%%%%%%%%%%%%%%%%%%%%%%%%%%%%%%%%%%%%%%%%%%%%%%%%%%%%%%%%%%%%%%%%%%%%%%%%%%%%%%%

      %%%%%%%%%%%%%%%%%%%%%%%%%%%%%%%%%%%%%%%%%%%%%%%%%%%%%%%%%%%%%%%%%%%%%%%%%%%%%%%%%%%%%%%%%%%%%%%%%%%%%%%%%%%%
      \begin{block}{Submit Script Example : MPI}
              \vspace*{-3ex}
        \begin{verbatim}
#!/bin/bash
#SBATCH -J MPI_JOB
#SBATCH -A uoa99999         # Project Account
#SBATCH --time=01:00:00     # Walltime
#SBATCH --ntasks=2          # number of tasks
#SBATCH --mem-per-cpu=8132  # memory/cpu (in MB)
#SBATCH -C sb               # sb=Sandybridge,wm=Westmere
srun mpi_binary
        \end{verbatim}
                \vspace*{-4ex}
      \end{block}
      %%%%%%%%%%%%%%%%%%%%%%%%%%%%%%%%%%%%%%%%%%%%%%%%%%%%%%%%%%%%%%%%%%%%%%%%%%%%%%%%%%%%%%%%%%%%%%%%%%%%%%%%%%%%

      %%%%%%%%%%%%%%%%%%%%%%%%%%%%%%%%%%%%%%%%%%%%%%%%%%%%%%%%%%%%%%%%%%%%%%%%%%%%%%%%%%%%%%%%%%%%%%%%%%%%%%%%%%%%
      \begin{block}{Submit Script Example : Hybrid (MPI+OpenMP)}
              \vspace*{-3ex}
        \begin{verbatim}
#!/bin/bash
#SBATCH -J Hybrid_JOB
#SBATCH -A uoa99999         # Project Account
#SBATCH --time=01:00:00     # Walltime
#SBATCH --ntasks=4          # number of tasks
#SBATCH --mem-per-cpu=8132  # memory/cpu (in MB)
#SBATCH --cpus-per-task=8   # 8 OpenMP Threads
#SBATCH --nodes=1           # number nodes
#SBATCH -C sb               # sb=Sandybridge,wm=Westmere
srun binary_hybrid
        \end{verbatim}
                \vspace*{-4ex}
      \end{block}
      %%%%%%%%%%%%%%%%%%%%%%%%%%%%%%%%%%%%%%%%%%%%%%%%%%%%%%%%%%%%%%%%%%%%%%%%%%%%%%%%%%%%%%%%%%%%%%%%%%%%%%%%%%%%

      %%%%%%%%%%%%%%%%%%%%%%%%%%%%%%%%%%%%%%%%%%%%%%%%%%%%%%%%%%%%%%%%%%%%%%%%%%%%%%%%%%%%%%%%%%%%%%%%%%%%%%%%%%%%
      \begin{block}{Submit Script Example : Hybrid (MPI+CUDA)}
              \vspace*{-3ex}
        \begin{verbatim}
#!/bin/bash
#SBATCH -J GPU_JOB
#SBATCH --time=01:00:00     # Walltime
#SBATCH -A uoa99999         # Project Account
#SBATCH --ntasks=4          # number of tasks
#SBATCH --ntasks-per-node=2 # number of tasks per node
#SBATCH --mem-per-cpu=8132  # memory/cpu (in MB)
#SBATCH --cpus-per-task=4   # 4 OpenMP Threads
# The following line will request GPUs per node. In this  
# particular example, it means 4 GPUs in total.
#SBATCH --gres=gpu:2       
#SBATCH -C kepler
srun binary_cuda_mpi
        \end{verbatim}
                \vspace*{-4ex}
      \end{block}
      %%%%%%%%%%%%%%%%%%%%%%%%%%%%%%%%%%%%%%%%%%%%%%%%%%%%%%%%%%%%%%%%%%%%%%%%%%%%%%%%%%%%%%%%%%%%%%%%%%%%%%%%%%%%


    \end{column}

%%%%%%%%%%%%%%%%%%%%%%%%%%%%%%%%%%%%%%%%%%%%%%%%%%%%%%%%%%%%%%
%%%%%%%%%%%%%%%%%%%%%%%%%%%%%%%%%%%%%%%%%%%%%%%%%%%%%%%%%%%%%%
%%%%%%%%%%%%%%%%%%%%%%%%%%%%%%%%%%%%%%%%%%%%%%%%%%%%%%%%%%%%%%
    
    \begin{column}{.35\linewidth}
    
      %%%%%%%%%%%%%%%%%%%%%%%%%%%%%%%%%%%%%%%%%%%%%%%%%%%%%%%%%%%%%%%%%%%%%%%%%%%%%%%%%%%%%%%%%%%%%%%%%%%%%%%%%%%%
      \begin{block}{Submit Script Example : Job Array}
              \vspace*{-3ex}
        \begin{verbatim}
#!/bin/bash
#SBATCH -J JobArray
#SBATCH --time=01:00:00     # Walltime
#SBATCH -A uoa99999         # Project Account
#SBATCH --ntasks=1          # number of tasks
#SBATCH --mem-per-cpu=8132  # memory/cpu (in MB)
#SBATCH --cpus-per-task=4   # 4 OpenMP Threads
#SBATCH --array=1-1000      # Array definition
#SBATCH -C sb               # sb=Sandybridge,wm=Westmere
srun binary_array $SLURM_ARRAY_TASK_ID
        \end{verbatim}
                \vspace*{-4ex}
      \end{block}
      %%%%%%%%%%%%%%%%%%%%%%%%%%%%%%%%%%%%%%%%%%%%%%%%%%%%%%%%%%%%%%%%%%%%%%%%%%%%%%%%%%%%%%%%%%%%%%%%%%%%%%%%%%%%


        %%%%%%%%%%%%%%%%%%%%%%%%%%%%%%%%%%%%%%%%%%%%%%%%%%%%%%%%%%%%%%%%%%%%%%%%%%%%%%%%%%%%%%%%%%%%%%%%%%%%%%%%%%%%
      \begin{block}{User Environment}
      \textbf{LMOD} is very useful to manage environment variables for each application and it is very easy to use. It loads the needed environment by a certain application and its dependencies automatically. The command line is fully compatible with the previous \textbf{Environment Modules}, and it provides simple short-cuts and advanced features.\\
      Syntax : \verb|module [options] sub-command [args ...]|\\
      \textbf{Loading/Unloading sub-commands}
        \begin{itemize}
        \item  \textbf{load $|$ add} load module(s)
        \item  \textbf{del $|$ unload} Remove module(s), do not complain if not found
        \item  \textbf{purge} unload all modules
        \item  \textbf{update} reload all currently loaded modules.
        \end{itemize}
       \textbf{Listing / Searching sub-commands}
        \begin{itemize}
        \item  \textbf{list} List loaded modules
        \item  \textbf{avail $|$ av} List available modules
        \item  \textbf{avail $|$ av string} List available modules that contain "string".
        \item  \textbf{spider} List all possible modules
        \item  \textbf{spider module} List all possible version of that module file
        \item  \textbf{spiderstring} List all module that contain the "string".
        \end{itemize}
       \textbf{Short-cuts}
       \begin{itemize}
        \item \textbf{ml} - means: module list
        \item \textbf{ml foo bar} - means: module load foo bar
        \item \textbf{ml -foo -bar baz goo}- means: module unload foo bar; module load baz goo;
        \end{itemize}
        More information at \url{http://www.tacc.utexas.edu/tacc-projects/lmod}
      \end{block}
      %%%%%%%%%%%%%%%%%%%%%%%%%%%%%%%%%%%%%%%%%%%%%%%%%%%%%%%%%%%%%%%%%%%%%%%%%%%%%%%%%%%%%%%%%%%%%%%%%%%%%%%%%%%%
      
    \end{column}
  \end{columns}
\end{frame}

\end{document}


%%%%%%%%%%%%%%%%%%%%%%%%%%%%%%%%%%%%%%%%%%%%%%%%%%%%%%%%%%%%%%%%%%%%%%%%%%%%%%%%%%%%%%%%%%%%%%%%%%%%
%%% Local Variables: 
%%% mode: latex
%%% TeX-PDF-mode: t
%%% End: 
